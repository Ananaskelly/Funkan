\documentclass[12pt,a4paper,titlepage]{book}
\usepackage[utf8]{inputenc}
\usepackage[russian]{babel}
\usepackage[OT1]{fontenc}
\usepackage{amsmath}
\usepackage{amsthm}
\usepackage{indentfirst}
\usepackage{amsfonts}
\usepackage{amssymb}
\usepackage[left=2cm,right=2cm,top=2cm,bottom=2cm]{geometry}
\title{Функциональный анализ}

\newcommand{\overbar}[1]{\mkern 1.5mu\overline{\mkern-1.5mu#1\mkern-1.5mu}\mkern 1.5mu}

\theoremstyle{definition}
\newtheorem*{definition}{Определение}

\theoremstyle{plain}
\newtheorem*{theorem}{Теорема}

\theoremstyle{remark}
\newtheorem*{remark}{Замечание}

\theoremstyle{remark}
\newtheorem*{example}{Пример}

\theoremstyle{remark}
\newtheorem*{cexample}{Контр-пример}

\theoremstyle{plain}
\newtheorem*{lemma}{Лемма}

\begin{document}

\chapter{Нормированные пространства}

Изучение свойств отображений, как и в математическом анализе, начнём с введения определений, связанных с областями задания отображений.\\

\section{Метрические пространства}

\begin{definition}В метрическом пространстве $X$ для любых элементов $x,y\in X$ определено расстояние $\rho(x, y)$, которое удовлетворяет требованиям (аксиомам метрического пространства):

\begin{enumerate}

	\item $\rho(x, y)\geqslant 0$, а $\rho(x, y)=0$ означает, что элементы $x$ и $y$ совпадают,

	\item $\rho(x, y)=\rho(y, x)$,

	\item $ \rho(x, y)\leqslant \rho(x, z) +\rho(z, y)$ --- неравенство треугольника.
	
\end{enumerate} 

\end{definition}

Расстояние (метрика) $\rho$ определяет сходимость последовательности $\lbrace x_n \rbrace_{n=1}^{\infty}  \in X$ к элементу $x^{*}\in X$:

\begin{center}

$x_n\rightarrow x^{*}$, если $\rho(x_n, x^{*})\rightarrow 0$ при $n\rightarrow\infty$

\end{center}

Из аксиом метрического пространства следует непрерывность функции $\rho(x, y)$, то есть если $x_n\rightarrow x^{*}$, $y_n\rightarrow y^{*}$ при $n\rightarrow\infty$, то $\rho(x_n, y_n)\rightarrow \rho(x^*, y^*)$.\\

Естественным образом вводятся понятия:

\begin{itemize}

	\item  открытый шар, замкнутый шар, окрестность элемента $x_0\in X$,

	\item внутренняя точка множества $M \in X$, открытое множество, замкнутое множество,

	\item подпространство $X_0$ метрического пространства: метрика в $X_0$ определяется метрикой пространства $X$, множество $X_0$ - замкнуто,

	\item фундаментальная последовательность $\lbrace x_n \rbrace_{n=1}^{\infty}$ --- последовательность, такая что $\forall \varepsilon>0$ существует номер $n=n(\varepsilon)$ такой что $\rho(x_n, x_{n+m})<\varepsilon$ при $n>n(\varepsilon)$, $m\geq1$ (последовательность, сходящаяся в себе)

\end{itemize}

\subsection*{Основные типы метрических пространств и множеств}

\begin{enumerate}

	\item \textbf{Полное метрическое пространство} --- любая фундаментальная последовательность имеет предел, принадлежащий $X$ (в математическом анализе --- признак Коши сходимости числовой последовательности).

	\item Множество $D\in X$ \textbf{плотно в множестве} $ M_0 \in X$, если для каждого элемента $x_0 \in M_0$ и любого $\varepsilon>0$ найдётся элемент $z \in D$, такой что $\rho(x_0, z)<\varepsilon$($z=z(\varepsilon)$). Если множество $D$ плотно в $M_0$, то для любого элемента $x_0\in M_0$ существует последовательность элементов $\lbrace z_n \rbrace \in D$ таких, что $\rho(z_n, x_0)\rightarrow0$ при $n\rightarrow\infty$. Ясно, что $\overline{D}=M_0$.

	\item \textbf{Сепарабельное пространство $X$} --- в таком пространстве существует счётное всюду плотное множество $D$: $D=\lbrace x_1,x_2,\ldots,x_n,\ldots \rbrace$. Для любого элемента $x_0 \in X$ можно найти такой номер $n=n(x_0,\varepsilon)$, что $\rho(x_0, z_n)<\varepsilon$.

	\begin{example}	
	Пространство $C[a,b]$ непрерывных на $[a,b]$ функций сепарабельно. В математическом анализе это теорема Вейерштрасса: для каждой непрерывной функции $x_0(t)$ существует полином $P_n$ с рациональными коэффициентами такой что
	\end{example}
	
	\begin{center}
	$\rho(x_0, P_n)=\smash{\displaystyle\max_{t \in [a,b]}} |x_0(t)-P_n(t)| <\varepsilon$.
	\end{center}
	
	Множество таких полиномов счётно.
	
	\item \textbf{Компактное множество метрического пространства $X$}

	Множество $K$ компактно в $X$, если в любой подпоследовательности элементов $\lbrace x_n \rbrace_{n=1}^{\infty} \in K$ существует фундаментальная подпоследовательность $\lbrace x_{n_k} \rbrace$, $n_{k+1}>n_k$(т.е. последовательность $n_k$ возрастает).

\end{enumerate}


Компактность множеств играет важную роль при исследовании приближённых методов решения задач.

Если $X$ полное пространство, то существует предельный элемент этой фундаментальной подпоследовательности, но он может не принадлежать множеству $K$.

Определение компактного множества неконструктивно. Следующая теорема Хаусдорфа даёт эффективный критерий компактности множеств.

\begin{definition}
Говорят, что в множестве $M \in X$ существует \textbf{конечная $\varepsilon$-сеть$\lbrace x_1,x_2,\ldots,x_{N(\varepsilon)} \rbrace$}, если для любого элемента $x \in M$ можно указать элемент $x_n$ $\varepsilon$-сети, такой что

\begin{center}
 $\rho(x_n, x) <\varepsilon$, $n=n(x_0,\varepsilon)$.
\end{center}

\end{definition}

\section{Пространства первой и второй категории}

Множество $D$ всюду плотно в метрическом пространстве $X$, если для каждого элемента $x \in X$ и любого $\epsilon > 0$ существует элемент $z \in D$ такой что
$\rho(x,z) < \epsilon$.

Сформулируем утверждение: множество $E$ не является множеством всюду плотным в пространстве $X$.

\begin{definition}
Множество $E$ \textbf{не является всюду плотным} в $X$ (\textbf{нигде не плотным} в $X$), если в любом замкнутом шаре $\overline{V_r}(x)$ существует замкнутый шар, в котором нет элементов множества $E$.
\end{definition}

\begin{example}
На плоскости (в пространстве $R_2$) множество точек любой прямой - нигде не плотное в $R_2$ множество.
Рассмотрим множество прямых $l$, параллельных оси $x$ и пересекающих ось $y$ в точках с рациональными значениями координат $y_n$. Ясно, что множество таких прямых счётно, и каждая прямая $l_n$ этого множества есть множество нигде не плотное в пространстве $R_2$.
\end{example}

Будет ли множество точек $\bigcup\limits_{k=1}^\infty l_k$ совпадать со всем пространством $R_2$? Ответ отрицателен: прямые, пересекающие ось $y$ в точках с иррациональными значениями, не принадлежат $\bigcup\limits_{k=1}^\infty l_k$.

\begin{definition}
Множество $E$ называется множеством первой категории, если оно представимо в виде 
$$E=\bigcup\limits_{k=1}^\infty E_k$$, 
где все множества $E_k$ нигде не плотные в $X$. Если множество $E$ нельзя представить в виде счетного объединения нигде не плотных множеств, то $E$ называется множеством второй категории.
\end{definition}

В общем случае верна теорема:

\begin{theorem}[Луи Бэр, 1905 г.]
Полное метрическое пространство $X$ является множеством второй категории.
\end{theorem}

\begin{proof}
(От противного)

Предположим, что $X=\bigcup\limits_{k=1}^\infty E_k$, где все множества нигде не плотные. Пусть $\overline{V_{r_0}}(x_0)$ произвольный шар. Так как $E_1$ нигде не плотно в $X$, то в этом шаре существует шар $\overline{V_{r_1}}(x_1)$, в котором нет элементов множества $E_1$. Можно считать, что радиус этого шара $r_1 < \frac{1}{2}r_0$.

Множество $E_2$ нигде не плотно: в шаре $\overline{V_{r_1}}(x_1)$ существует шар $\overline{V_{r_2}}(x_2)$, в котором нет элементов множества $E_2$ (и элементов множества $E_1$). Можно считать, что $r_2 < \frac{1}{2}r_1 < \frac{1}{2^2}r_0$.

Продолжая этот процесс, получим последовательность вложенных шаров $\overline{V_{r_1}}(x_1) \supset \overline{V_{r_2}}(x_2) \supset$ ... $\supset \overline{V_{r_n}}(x_n) \supset ...; r_n < \frac{1}{2^n}r_0, r_n \rightarrow 0$ при $n \rightarrow \infty$, и в каждом шаре $\overline{V_{r_n}}(x_n)$ нет элементов множеств $E_1, E_2, ..., E_n$.

По теореме о вложенных шарах существует элемент $x^*\subset X$:
$x_n \rightarrow x^*$ при $n \rightarrow \infty$.

Ясно, что $x^*\in \bigcup\limits_{n=1}^\infty \overline{V_{r_n}}(x_n)$ и следовательно $x^*\notin \bigcup\limits_{n=1}^\infty E_n$, что противоречит предположению $X=\bigcup\limits_{n=1}^\infty E_n$. Остается принять, что множество $X$ пространство второй категории.

\end{proof}

\section{Линейные пространства, нормированные пространства, пространства Банаха}

Множество элементов $Х$ называется \textbf{линейным множеством}, если для его элементов определены действия сложения и умножения на число (вещественное или комплексное), не выводящие из множества $X$:

\begin{itemize}

	\item если $x,y\in X$, то $x+y\in X$

	\item если $x\in X$, то $\lambda x\in X$

\end{itemize}


Эти действия должны удовлетворять обычным условиям (аксиомам). Если $\lambda$ вещественные числа, то $X$ --- вещественное линейное множество, если $\lambda$ комплексные, то $X$ --- комплексное линейное множество. Для вещественного линейного множества $Х$ можно построить комплексное линейное множество $Z$: достаточно ввести элементы $z=x+iy,\quad x,y \in X$ и определить сумму элементов:

\begin{center}
$z_1+z_2=(x_1+x_2)+i(y_1+y_2)$,
\end{center}
и ввести умножение на комплексное число $\lambda$:
\begin{center}
$\lambda z = (\alpha + i\beta)(x+iy) = (\alpha x - \beta y)+i(\alpha y + \beta x)$
\end{center}
(комплексификация линейного множества $X$).

Для комплексного линейного множества $Z$ каждый элемент $z=x+iy$, где $x$ и $y$ --- элементы вещественного множества.

Рассмотрим вещественное пространство $X$ пар $(x,y)$, в котором определим сумму: $(x_1,y_1)+(x_2,y_2) = (x_1+x_2,y_1+y_2)$ и умножение на вещественное число $\lambda$: $\lambda(x,y) = (\lambda x, \lambda y)$. Множество пар $(x,y)$ образует вещественное линейное множество  $X$ (декомплексификация комплексного линейного множества $Z$).

Из аксиом линейного множества отметим некоторые следствия:

\begin{enumerate}

	\item Существования нулевого элемента: $\ominus=x-x=(1-1)x=0x$.

	\item Из равенства $\lambda x=0$ при $\lambda\ne0$ следует $x=\ominus$.

	\item Определение линейной независимости элементов $\{x_1,x_2,\dotsc,x_n\}$.

	\item Определение размерности линейного множества $X$ как наибольшего числа линейно независимых элементов множества $X$.

	\item Линейное множество бесконечномерно, если для любого натурального $n$ существует $n$ линейно независимых элементов.

\end{enumerate}

\subsection*{Примеры линейных множеств}

\begin{enumerate}

	\item Вещественное пространство $V_n$ $n$-мерных векторов.

	\item Множество прямоугольных матриц размерности $(n\times m)$.

	\item Множество $C[t_0,t_1]$ непрерывных на $[t_0,t_1]$ функций $x$. Функции $x_k(t)=t^k,\; k=1,2,3,\dotsc$ линейно независимы, а пространство $C[t_0,t_k]$ бесконечномерно.

	\item Множество решений $x\in C^n[t_0,t_1]$ уравнения

	\begin{center}
	$\frac{d^nx(t)}{dt^n}+a_1\frac{d^{n-1}x(t)}{dt^{n-1}}+\dotsb+a_{n-1}\frac{dx(t)}{dt}+a_n=0$, где $a_k\in C[t_0,t_1]$
	\end{center}

\end{enumerate}


Снабжая линейное пространство метрикой, мы получаем более богатую теорию. Связь метрики с алгебраическими действиями реализуется введением норм элементов $x$: норма $\lVert x\rVert$ элемента $x\in X$, согласно определению есть число, которое должно удовлетворять трем условиям:

\begin{enumerate}

	\item $\lVert x\rVert \geq 0$; если $\lVert x\rVert = 0$, то $x=\ominus$.

	\item $\lVert \lambda x \rVert = \lvert \lambda \rvert \lVert x \rVert$.

	\item $\lVert x+y \rVert\leq\lVert x \rVert + \lVert y \rVert$.

\end{enumerate}


Норма $\lVert x\rVert$ является непрерывной функцией: $\lvert\lVert x+\triangle x\rVert-\lVert x\rVert\rvert\to0$ при $\lVert\triangle x\rVert\to0$.


Верно неравенство $\lVert x-y\rVert\geq\lvert\lVert x\rVert-\lVert y\rVert\rvert$.


Определим метрику в линейном пространстве $X$: $\rho(x,y)=\lVert x-y\rVert$. Ясно, что введенная таким образом метрика удовлетворяет всем аксиомам метрического пространства. Линейное множество с метрикой, определяемой нормой элементов, называется \textbf{нормированным пространством}. Если нормированное пространство полное, то оно называется \textbf{пространством Банаха} (Стефан Банах, 1892-1945, польский математик), банаховым пространством, В-пространством.

\begin{definition}
\textbf{Подпространством нормированного пространства $X$} называется любое линейное \underline{замкнутое} множество $X_0 \in X$.
\end{definition}

\textbf{Примеры}

\begin{enumerate}

	\item Банаховы пространства $n$-мерных векторов получаем введением различных норм векторов $\bar{x}(x_1,x_2,\dots,x_n)$:

	\begin{itemize}

		\item $\lVert \bar{x}\rVert_\infty=\displaystyle\max_{i}\lvert x_i\rvert$,

		\item $\lVert \bar{x}\rVert_1 = \displaystyle\sum_{i}\lvert x_i\rvert$,

		\item $\lVert \bar{x}\rVert_2 = (\displaystyle\sum_{i}\lvert x_i\rvert^2)^{\frac{1}{2}}$

	\end{itemize}

	\item Бесконечномерное банахово пространство $C[t_0,t_1]$ функций $x(t)$ непрерывных на $[t_0,t_1]$. Норма:

	\begin{center}
	$\lVert x\rVert = \displaystyle\max_{i}\lvert x\rvert$,
	\end{center}

	функции $x_k(t)=t^k, k=1,2,3,\dots$ линейно независимы.

	\item Бесконечномерное пространство банахово пространство $C_n[t_0,t_1]$. Норма:

	\begin{center}
	$\lVert x\rVert = \displaystyle\sum_{k=0}^n\max_{i}\lvert\frac{d^kx(t)}{dt^k}\rvert$
	\end{center}

	\item Пространство Банаха $L_p(a,b)$ измеримых и суммируемых со степенью $p,\;p\geq1,$ функций. Норма:

	\begin{center}
	$\lVert x\rVert^p = (\int\limits_a^b\lvert x(t)\rvert^pdt)^{\frac{1}{p}}$
	\end{center}

	Множество полиномов с комплексными коэффициентами плотно в этих пространствах.

	\item Пример неполного нормированного пространства.

	В линейном множестве $C[0,1]$ непрерывных функций введем норму (и метрику):

	\begin{center}
	$\lVert x\rVert = (\int\limits_a^b\lvert x(t)\rvert^pdt)^{\frac{1}{p}},\quad\rho(x,y) = (\int\limits_0^1\lvert x(t)-y(t)\rvert^pdt)^{\frac{1}{p}}$
	\end{center}

	Получаемое пространство не является полным. Действительно, последовательность функций $x_k(t)=t^k$ является фундаментальной последовательностью:

	\begin{center}
	$\lVert x_{n+m}-x_n\rVert^p=\int\limits_0^1(t^n-t^{n+m})^pdt=\int\limits_0^1t^{np}(1-t^m)^pdt<\int\limits_0^1t^{np}dt=\frac{1}{np+1}\to0$, при $n\to\infty$
	\end{center}

	Предел же $\lim x_n(t)$ при $n\to\infty$ в пространстве $C[0,1]$ не существует.

\end{enumerate}

\section{Пространства Гильберта}

Рассматривается линейное комплексное пространство, в котором введено скалярное произведение $(x, y)$ элементов $x$ и $y$, удовлетворяющее обычным свойствам скалярного произведения:

\begin{enumerate}

    \item $(x, y) = \overbar{(x, y)}$

    \item $(\lambda x_1 + \mu x_2, y) = \lambda (x_1, y) + \mu (x_2, y)$

    \item $(x, x)$ - вещественное число, $(x, x) \geqslant 0$ и если $(x, x) = 0$, то $x = \ominus$

\end{enumerate}

Верно неравенство Коши-Буняковского:

$$|(x, y)|^2 \le (x, x) \cdot (y, y)$$
Действительно, 
$$(x + \lambda y, x + \lambda y) \ge 0$$
$$(x, x) + \lambda (y, x) + (x, \lambda y) + \lambda \overbar{\lambda} (y, y) > 0$$
$$(x,x) + 2 Re(\lambda y, x) + |\lambda|^2 (y, y) \ge 0$$
для любых чисел $\lambda$.




%стр.18 п.4 - стр.23
\begin{enumerate}
\item Свойство элементов $y_j$.

$\lVert By_j\rVert = \lVert B(u_j - u_0)\rVert$ (аддитивность $B$) $ = \lVert Bu_j - Bu_0\rVert \le \lVert Bu_j\rVert + \lVert Bu_0\rVert(u_0, u_j \in Y_n) \le n(\lVert u_j\rVert + \lVert u_0\rVert)\cdot 1$.
Оценим $\lVert B_j\rVert$ через норму элементов $\lVert y_j\rVert$. Так как $y_j \to y$, то $\lVert y_j\rVert \to \lVert y\rVert=r_0$, и при достаточно больших значениях $j$:

\begin{center}
$\lVert y_j\rVert\frac{1}{r_0}>\frac12$ и $1<\frac{2}{r_0}\lVert y_j\rVert$.
\end{center}

Для таких значений $j$: $\lVert By_j\rVert\le\frac{2}{r_0}n(\lVert u_j\rVert + \lVert u_0\rVert)$ и так как $\lVert u_j\rVert = \lVert u_0 + u_j - u_0\rVert \le \lVert u_0\rVert + \lVert u_j - u_0\rVert \le \lVert u_0\rVert + r_0$, то для достаточно больших значений $j$:

\begin{center}
$\lVert By_j\rVert_X \le \frac{2}{r_0}(r_0 + \lVert u_0\rVert)\cdot n\cdot \lVert y_j\rVert_Y$.
\end{center}

Величину $\frac{2}{r_0}(r_0 + \lVert u_0\rVert)\cdot n$ оценим натуральным числом $N$:

\begin{center}
$\lVert By_j\rVert \le N\lVert y_j\rVert$ и $y_j \in Y_N$.
\end{center}

Так как $y_j \to y$, то множество $Y_N$ плотно в множестве элементов $y$ с нормой $r_0$. Значение $N$, согласно п.п. 1.2, зависит только от фиксированных значений $r$, $n_0$ и $u_0 \in Y$. Так как множество $Y_N$ ``однородно'', то и для любого $y \in Y$ существует последовательность элементов $y_j \in Y_N$, такая что $y_j \to y$ при $j \to \infty$ и $\lVert By_j\rVert \le N\lvert y_j\rVert$. Обозначив $M = Y_N$ и $C_0 = N$, завершим доказательство леммы.

\end{enumerate}


\begin{theorem}[Банаха]
Замкнутый оператор $B$, действующий из банахова пространства $Y$ в банахово пространство $X$ и определённый на всём пространстве $Y$,линеен.
\end{theorem}

\begin{proof}
Согласно лемме в пространстве $Y$ существует всюду плотное множество $M$, для элементов которого
\begin{center}
	$\parallel By \parallel \leq C_0 \parallel y \parallel $
\end{center}
\begin{enumerate}

	\item Пусть $y_0$ любой элемент пространства $Y$.\\
	Построим шар радиуса $\frac{1}{4}\parallel y_0 \parallel$ с 		центром $\frac{3}{4}y_0$. В этом шаре найдём элемент
	$y_1 \in M$:
	\begin{center}
	$\parallel y_1 - \frac{3}{4}y_0\parallel \leq \frac{1}{4}    		\parallel y_0 \parallel $  ,
	\end{center}
	\begin{center}
	$\parallel y_1 \parallel =
	\parallel y_1 - \frac{3}{4}y_0 +
	\frac{3}{4}y_0 \parallel \leq \frac{1}{4} 			\parallel y_0 \parallel +
	 \frac{3}{4}\parallel y_0\parallel = 				\parallel y_0 \parallel$
	\end{center}
	
	\item Рассмотрим элемент $y_1 - y_0$. Для него
	\begin{center}
	$\parallel y_1-y_0 \parallel =\parallel y - \frac{3}{4}y_0 +
	\frac{1}{4}y_0 \parallel \leq
	\frac{1}{4} \parallel y_0 \parallel +
	\frac{1}{4} \parallel y_0 \parallel =
	\frac{1}{2} \parallel y_0 \parallel$
	\end{center}
	Построим шар радиуса $\frac{1}{4} \parallel 		y_1 - y_0 \parallel$ с центром $\frac{3}{4}			(y_1 - y_0)$. В этом шаре найдём элемент 			$y_2 \in M$:
	\begin{center}
	$\parallel y_2 + y_1 - y_0 \parallel =
	\parallel y_2 - \frac{3}{4}(y_1 - y_0) -
	\frac{1}{4}(y_1 - y_0) \parallel \leq
	\frac{1}{4} \parallel y_1 - y_0 \parallel +
	\frac{1}{4} \parallel y_1 - y_0 \parallel = $
	\end{center}
	\begin{center}
	$ = \frac{1}{2} \parallel y_1 - y_0 \parallel 			\leq
	\frac{1}{2^2} \parallel y_0 \parallel$ ,
	\end{center}
	\begin{center}
	$\parallel y_2 \parallel =
	\parallel y_2 - \frac{3}{4}(y_1 - y_0) +
	\frac{3}{4}(y_1 - y_0) \parallel \leq
	\frac{1}{4} \parallel y_1 - y_0 \parallel +
	\frac{3}{4} \parallel y_1 - y_0 \parallel =
	 \parallel y_1 - y_0 \parallel \leq$
	 \end{center}
	\begin{center}
	 $ \leq \frac{1}{2} \parallel y_0 \parallel$
	\end{center}
	
	\item Рассмотрим элемент $y_2+y_1- y_0$ . Для него 	\begin{center}	
	$\parallel y_2+y_1- y_0 \parallel \leq
	\frac{1}{2^2} \parallel y_0 \parallel$ .
	\end{center}
	Построим шар радиуса
	$\frac{1}{4} \parallel y_2 + y_1 - y_0 \parallel$      	с центром $\frac{3}{4}(y_2 + y_1 - y_0)$.
	В этом шаре найдём элемент 	$y_3 \in M$:
	\begin{center}
	$\parallel y_3 + y_2 + y_1 - y_0 \parallel =
	\parallel y_3 - \frac{3}{4}(y_0 - y_1 - y_2) -
	\frac{1}{4}(y_0 - y_1 - y_2) \parallel \leq $
	\end{center}
	\begin{center}
	$\leq \frac{1}{4} \parallel y_0 - y_1 - y_2 \parallel +
	\frac{1}{4} \parallel y_0 - y_1 - y_2 \parallel 			\leq $
	\end{center}
	\begin{center}
	$ \leq\frac{1}{2} \cdot
	\frac{1}{2^2} \parallel y_0 \parallel =
	\frac{1}{2^3} \parallel y_0 \parallel$ ,
	\end{center}
	\begin{center}
	$\parallel y_3 \parallel =
	\parallel y_3 - \frac{3}{4}(y_0 - y_1 - y_2) +
	\frac{3}{4}(y_0 - y_1 - y_2) \parallel \leq $
	\end{center}
	\begin{center}
	$\leq \frac{1}{4} \parallel y_0 - y_1 - y_2 				\parallel +
	\frac{3}{4} \parallel y_0 - y_1 - y_2 \parallel =
	 \parallel y_0 - y_1 - y_2 \parallel \leq
	 \frac{1}{2^2} \parallel y_0 \parallel$
	\end{center}

Продолжая этот процесс, получим элементы $y_n, y_{n-1}, y_{n-2}, ..., y_3, y_2, y_1 \in M$ такие что 
\begin{center}
$\lVert y_n \rVert \leq \frac{1}{2^{n-1}} \lVert y_0 \rVert$\\
и\\
$\lVert y_n + y_{n-1}+y_{n-2}+...+y_2+y_1+y_0 \rVert \leq \frac{1}{2^n} \lVert y_0 \rVert.$
\end{center}

Обозначим $s_n= \sum\limits_{k=1}^n y_k$. Тогда $s_n \rightarrow y_0$ при $n \rightarrow \infty$. Последовательность элементов $B s_n$ сходится в себе:
\begin{center}
$ \lVert B s_{n+m}-B s_n \rVert _X = \lVert B(s_{n+m}-s_n) \rVert= \lVert B(y_{n+1}+y_{n+2}+...+y_{n+m}) \rVert \leq $
\end{center}
\begin{center}
$\leq C_0 \lVert y_0 \rVert (\frac{1}{2^n} + \frac{1}{2^{n+1}}+...+\frac{1}{2^{n+m-1}})=C_0 \frac{1}{2^n} (1 + \frac{1}{2}+...+\frac{1}{2^{m-1}}) \rightarrow 0$ при $n \rightarrow \infty$.
\end{center}

Так как пространство $X$ полное, то существует $\lim \limits_{n \rightarrow \infty} s_n=x^* \in X$. \\
Переходя в оценке $ \lVert B s_{n+m}-B s_n \rVert$ к пределу при $m \rightarrow \infty$, получаем
\begin{center}
$ \lVert x^*-B s_n \rVert \leq C_0 \lVert y_0\rVert (\frac{1}{2^n} + \frac{1}{2^{n+1}}+...)$.
\end{center}

Так как $s_n \rightarrow y_0, B s_n \rightarrow x^*$, то в силу замкнутости оператора $B$: 
\begin{center}
$B y_0=x^* $.
\end{center}

Оценим $\lVert B y_0 \rVert _X$:
\begin{center}
$\lVert B y_0 \rVert \leq \lVert x^* - B s_n \rVert + \lVert B s_n \rVert \leq$ 
\end{center}
\begin{center}
$\leq C_0 \lVert y_0\rVert [\frac{1}{2^n} + \frac{1}{2^{n+1}}+...] + C_0 \lVert y_0\rVert [1+\frac{1}{2} +\frac{1}{2^2}+ ...+\frac{1}{2^{n-1}}]= 2 C_0 \lVert y_0\rVert$
\end{center}
$\lVert B y_0 \rVert \leq 2 C_0 \lVert y_0\rVert$, т.е. оператор $B$ линеен, $ \lVert B y_0 \rVert _{Y \rightarrow X} \leq 2 C_0$.
\end{enumerate}
\end{proof}



\section{Вполне непрерывные операторы}
\begin{definition}Линейный оператор $A \in Z(X, Y)$ называется \textbf{вполне непрерывным}, если любое ограниченное в $X$ множество он отображает в множество, компактное в $Y$.
\end{definition}
Напомню, что в компактном множестве в любой последовательности $\lbrace y_n \rbrace_{n=1}^{\infty}$ содержится фундаментальная последовательность. Если же пространство $Y$ полно, то согласно определению эта фундаментальная последовательность имеет предел в $Y$.

Множество всех вполне непрерывных операторов обозначим $\sigma(X,Y)$.
\begin{theorem}Множество $\sigma(X,Y)$ является подпространством пространства $Z(X, Y)$.
\end{theorem}
\begin{proof}Состроит в доказательстве двух пунктов (согласно определению подпространства).

I. Если $A_1, A_2 \in \sigma(X,Y)$, то их линейная комбинация $A=\lambda_1 A_1+\lambda_2 A_2\in \sigma(X,Y)$.\\
Рассмотрим множество $AS_1$, где $S_1$ --- единичная сфера в пространстве $X$. Покажем, что множество $AS_1$ компактно в $Y$. Возьмем любую последовательность элементов $x_n \in S_1$, $\lVert x_n \rVert=1$. Обозначим элементы $Ax_n=y_n$, $y_n=\lambda_1 A_1 x_n+\lambda_2 A_2 x_n$.

Так как множество $A_1S_1$ компактно в $Y$, то из последовательности $\lbrace A_2x_{n_k} \rbrace$ можно выделить фундаментальную последовательность $\lbrace A_2x'_i \rbrace$. Ясно, что последовательность $\lbrace (\lambda_1 A_1+\lambda_2 A_2)x'_i \rbrace$ является фундаментальной последовательностью в $Y$.

II. Покажем, что множество $\sigma(X,Y)$ замкнуто. Так как $\lVert A_n-A \rVert \rightarrow 0$ при $n\rightarrow \infty$, то для выбранного $\varepsilon > 0$ рассмотрим операторы $A_n$ такие что $\lVert A_n-A \rVert \leq  \varepsilon$ и при $x \in S_1$ $\lVert A_n x-Ax\rVert \leq  \varepsilon$. Зафиксируем $n$. Рассмотрим множество элементов $A_n S_1$. Так как множество $A_n S_1$ компактно в $Y$, то в множестве $A_n S_1$ существует конечная $\varepsilon$-сеть $\lbrace y_k \rbrace$ $y_k=A_n x_k$, $x_k \in S_1$. Тогда $\lVert y_k - Ax\rVert \leq \lVert y_k - A_n x\rVert + \lVert A_n x - Ax\rVert \leq 2\varepsilon$. Следовательно, элементы $y_k$ образуют $2\varepsilon$-сеть в множестве $Y$ и, согласно теореме Хаусдорфа, множество $AS_1$ компактно.
\end{proof}

\begin{cexample}Тождественный оператор $E$ в сепарабельном гильбертовом пространстве не является вполне непрерывным оператором.

Действительно, пусть $\psi_1,\psi_2,...,\psi_n,..., \lVert\psi_n\rVert =1$ --- ортонормальный базис пространства. Множество $S_1$ ограничено, но множество $ES_1 (=S_1)$ не является компактным: из последовательности $\lbrace \psi_n \rbrace_{n=1}^{\infty}$ нельзя выбрать фундаментальную последовательность, так как $\lVert\psi_n - \psi_m \rVert ^2=(\psi_n - \psi_m, \psi_n - \psi_m)= (\psi_n,\psi_n)+ (\psi_m,\psi_m)=2$ при $n \neq m$
\end{cexample}
\begin{example}Интегральный оператор $\widetilde{K}$ из $L_2(a,b)$ в $L_2(a,b)$.
\begin{center}
$y=\widetilde{K}x$, $y(t)=\int\limits_a^b \widetilde{K}(t, \tau)x(\tau)d\tau ,$\\
\end{center}
где ядро $\widetilde{K}(t, \tau)$ непрерывно в области $D=[a,b]\times[a,b]$, $|\widetilde{K}(t, \tau)|\leq M$.\\

В этом случае функции $y(t)$ непрерывны:

\begin{center}
$|y(t_2)-y(t_1)|^2\leq \int\limits_a^b |\widetilde{K}(t_2, \tau)-\widetilde{K}(t_1, \tau)|^2d\tau \cdot \int\limits_a^b |x(\tau)|^2 d\tau \leq \varepsilon^2 (b-a){\lVert x\rVert^2}_{L_2(a,b)}$,\\
\end{center}
Величина $|y(t_2)-y(t_1)|\rightarrow 0$ при $|t_2-t_1|\rightarrow 0$ в силу непрерывности функции $\widetilde{K}(t, \tau)$ как функции двух переменных. Таким образом, множество функций $\widetilde{K} S_1$ равностепенно непрерывно.

Ясно, что функции множества $\widetilde{K} S_1$ ограничены в совокупности: $|y(t)|^2 \leq M^2 (b-a)$.

По теореме Арцела-Асколи множество $\widetilde{K} S_1$ компактно в $C[a,b]$: из любой последовательности элементов $y_n=\widetilde{K} x_n$ можно выделить фундаментальную последовательность в $C[a,b]$, которая является фундаментальной последовательностью и в пространстве $L_2 (a,b))$.
\end{example}
\begin{example}Интегральный оператор $K$ из $L_2(a,b)$ в $L_2(a,b)$.
\begin{center}
$y=Kx$, $y(t)=\int\limits_a^b K(t, \tau)x(\tau)d\tau $,
\end{center}
где $K(t, \tau)\in L_2(D)$ (интегральный оператор Гильберта-Шмидта).

По теореме Лебега для функции $K(t, \tau)$ существует последовательность непрерывных в $D$ функций $\widetilde{K}_n (t, \tau)$, таких что
\begin{center}
$\int\limits_a^b |K(t, \tau)-K_n(t, \tau)|^2d\tau dt \rightarrow 0$ при $n\rightarrow \infty$,
\end{center}
\begin{center}
т.е. $\lVert K-\widetilde{K}_n \rVert \rightarrow 0$ при $n\rightarrow \infty$.
\end{center}

Так как интегральные операторы $\widetilde{K}_n$ вполне непрерывны, то и интегральный оператор Гильберта-Шмидта вполне непрерывен.
\end{example}
\begin{theorem}Пусть оператор $A \in \sigma (H,H)$, где $H$ бесконечномерное сепарабельное пространство Гильберта. Задача решения уравнения $Ax=y$ поставлена некорректно по Адамару.
\end{theorem}
\begin{proof}В этом случае легко доказать, что нарушено условие непрерывной зависимости решения при вариации первой части. Действительно, так как множество $AS_1$ компактно в $H$, то из последовательности $\lbrace A \psi_n\rbrace_{n=1}^{\infty}$ можно выбрать сходящуюся подпоследовательность элементов $\lbrace A \psi_{n_k}\rbrace_{n=1}^{\infty}$, а так как пространство Гильберта полное, то эта фундаментальная последовательность сходится к элементу $y_0 \in H$: $y_0=\underset{k \rightarrow \infty}{lim} A \psi_{n_k}$.\\
Рассмотрим вариацию правой части $\Delta y_k=A \psi_{n_k}-y_0 \rightarrow \infty$ при $k\rightarrow \infty$. Соответствующая вариация решения $\Delta x_k=\psi_{n_k}$ и $\lVert\Delta x_k \rVert=1$, т.е. вариация решения $\Delta x_k$ не стремится к нулю при $\lVert\Delta y_k \rVert \rightarrow \infty$.

\end{proof}

\end{document}
