\documentclass[12pt,a4paper,titlepage]{book}
\usepackage[utf8]{inputenc}
\usepackage[russian]{babel}
\usepackage[OT1]{fontenc}
\usepackage{amsmath}
\usepackage{amsthm}
\usepackage{amsfonts}
\usepackage{amssymb}
\usepackage[left=2cm,right=2cm,top=2cm,bottom=2cm]{geometry}
\title{Функциональный анализ}

\newcommand{\overbar}[1]{\mkern 1.5mu\overline{\mkern-1.5mu#1\mkern-1.5mu}\mkern 1.5mu}

\theoremstyle{definition}
\newtheorem{definition}{Определение}

\theoremstyle{plain}
\newtheorem{theorem}{Теорема}

\theoremstyle{remark}
\newtheorem*{remark}{Замечание}

\theoremstyle{plain}
\newtheorem{lemma}{Лемма}

\begin{document}

\chapter{Нормированные пространства}

Изучение свойств отображений, как и в математическом анализе, начнём с введения определений, связанных с областями задания отображений.\\

\section{Метрические пространства}

\section{Пространства первой и второй категории}

\section{Линейные пространства, нормированные пространства, пространства Банаха}

\section{Пространства Гильберта}

Рассматривается линейное комплексное пространство, в котором введено скалярное произведение $(x, y)$ элементов $x$ и $y$, удовлетворяющее обычным свойствам скалярного произведения:

\begin{enumerate}

    \item $(x, y) = \overbar{(x, y)}$

    \item $(\lambda x_1 + \mu x_2, y) = \lambda (x_1, y) + \mu (x_2, y)$

    \item $(x, x)$ - вещественное число, $(x, x) \geqslant 0$ и если $(x, x) = 0$, то $x = 0$

\end{enumerate}

\end{document}