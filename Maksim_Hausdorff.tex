\documentclass[12pt,a4paper,titlepage]{book}
\usepackage[utf8]{inputenc}
\usepackage[russian]{babel}
\usepackage[OT1]{fontenc}
\usepackage{amsmath}
\usepackage{amsthm}
\usepackage{ mathrsfs } %add this pls
\usepackage{indentfirst}
\usepackage{amsfonts}
\usepackage{amssymb}
\usepackage[left=2cm,right=2cm,top=2cm,bottom=2cm]{geometry}
\title{Функциональный анализ} 
\newcommand{\overbar}[1]{\mkern 1.5mu\overline{\mkern-1.5mu#1\mkern-1.5mu}\mkern 1.5mu}

\theoremstyle{definition}
\newtheorem*{definition}{Определение}

\theoremstyle{plain}
\newtheorem*{theorem}{Теорема}

\theoremstyle{plain}
\newtheorem*{corollary}{Следствие}

\theoremstyle{remark}
\newtheorem*{remark}{Замечание}

\theoremstyle{remark}
\newtheorem*{example}{Пример}

\theoremstyle{plain}
\newtheorem*{lemma}{Лемма}



\begin{document}
\begin{theorem}[Хаусдорф, около 1914 г.]
Для того, чтобы множество $K \subset X$ было компактно в $X$, необходимо и достаточно чтобы для любого $\varepsilon > 0$ в множестве $K$ существовала конечная $\varepsilon$ \-- сеть. 
\end{theorem}
\begin{proof}

$\underbar{Необходимость}$ \textit{(Доказывается от противного)}.

Пусть $K$  \-- компактное в $X$ множество. Предположим, что для заданного $\varepsilon >0 $ не существует конечной $\varepsilon$ \-- сети. Возьмем любой элемент $x_1 \in K.$ Согласно предположению он не образует конечной $\varepsilon$ \-- сети и существует элемент $x_2 \in K$ такой что $\rho(x_1, x_2) > \varepsilon.$ Два элемента $x_1$ и $x_2$ не образуют $\varepsilon$ \-- сети, и существует третий элемент $x_3 \in K$ такой что значения $\rho(x_3, x_1), \; \rho(x_3, x_2), \; \rho(x_2, x_1) > \varepsilon.$

Продолжая этот процесс, получим последовательность элементов $x_1,x_2,x_3, \dots x_n\dots \in K$ таких что $\rho(x_i, x_j) > \varepsilon$ при   $i \neq j.$ Из этой последовательности нельзя составить ни одной фундаментальной подпоследовательности, что противоречит компактности множества $K.$

$\underbar{Достаточность}$ Пусть $\{x_n\}_{n=1}^{\infty}$ любая последовательность элементов множества $K.$ Образуем последовательность чисел $\varepsilon _k > 0,$ монотонно стремящуюся к $0.$

Для значения $\varepsilon _1$  в множестве $K$ существует конечная $\varepsilon _1$ \-- сеть, т.е. все множество $K$ может быть покрыто конечным числом шаров радиуса $\varepsilon _1 .$ Так как последовательность $\{ x_n \}$ содержит бесконечное число элементов, то среди упомянутых шаров найдется хотя бы один шар $V_{\varepsilon _1} (z_1),$ в котором содержится бесконечное число элементов последовательности $\{x_n\}.$ Обозначим $x_{n_1}$ первый из таких элементов: $x_{n_1} \in V_{\varepsilon _1} (z_1).$

Далее, шар $ V_{\varepsilon _1} (z_1) $ может быть покрыт конечным числом шаров радиуса $\varepsilon _2 < \varepsilon _1.$ Тогда в одном из таких шаров $V_{\varepsilon _2} (z_2)$ содержится бесконечное число элементов последовательности $\{x_n\},$ и первый после $x_{n_1}$ такой элемент обозначим $x_{n_2}:$
\begin{equation*}
x_{n_1} \in V_{\varepsilon _1} (z_1) \cap V_{\varepsilon _2} (z_2),
\end{equation*}
Продолжая этот процесс, получим последовательность элементов $\{x_{n_k}\} \subset \{x_n\}$ таких что $x_{n_k} \in V_{\varepsilon _k} (z_k),$
\begin{equation*}
x_{n_k} \in \bigcap\limits_{i=1}^{k} V_{\varepsilon _i} (z_i),
\end{equation*}
где $n_k$  возрастающая последовательность чисел. При $m > k$ оба элемента $x_{n_k}$ и $x_{n_m}$ принадлежат шару $V_{\varepsilon _k} (z_k).$ По неравенству треугольника 
\begin{equation*}
\rho(x_{n_k}, x_{n_m}) \leqslant \rho(x_{n_k}, z_k) + \rho(z_k, x_{n_m}) \leqslant \varepsilon_k + \varepsilon_k = 2 \varepsilon _k.
\end{equation*}
 
Следовательно подпоследовательность $\{x_{n_k}\}$ последовательности $\{x_n\}$ является фундаментальной.
\end{proof}

\begin{corollary}
Если в множестве $K \subset X$ существует компактная в $X$ $\varepsilon$ \-- сеть $H_{\varepsilon},$ то множество $K$ компактно в $X.$ 
\end{corollary}

Действительно, так как $H_{\varepsilon} $ является $\varepsilon$ \-- сетью для множества $K,$ то для любого элемента $x \in K$ существует элемент $x_{\varepsilon} \in H_{\varepsilon}$ такой что $\rho(x, x_{\varepsilon}) < \varepsilon.$ Из условия компактности множества $H_{\varepsilon}$ в пространстве $X$ следует, что в  $H_{\varepsilon}$ существует конечная $\varepsilon$ \-- сеть элементов $\bar{x}_1, \bar{x}_2,\dots \bar{x}_n,$ и для элемента $x_{\varepsilon}$ существует элемент $\bar{x}_k \in H_{\varepsilon}$ такой что $\rho(x_{\varepsilon}, \bar{x}_k) < \varepsilon.$ Тогда $\rho(x, \bar{x}_k) \leqslant \rho(x, x_{\varepsilon}) + \rho(x_{\varepsilon}, \bar{x}_k) \leqslant \varepsilon + \varepsilon = 2 \varepsilon.$

Следовательно элементы $\bar{x}_1, \bar{x}_2,\dots \bar{x}_n,$ множества $H_{\varepsilon}$ образуют в множестве $K$ конечную $2 \varepsilon$ \-- сеть. По теореме Хаусдорфа множество $K$ компактно в $X.$

Ясно, что компактное множество ограничено: существует шар конечного радиуса, которому принадлежит компактное множество.
\end{document}