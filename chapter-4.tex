\documentclass[12pt,a4paper,titlepage, oneside]{book}
\usepackage[utf8]{inputenc}
\usepackage[russian]{babel}
\usepackage[OT1]{fontenc}
\usepackage{amsmath}
\usepackage{amsthm}
\usepackage{mathrsfs}
\usepackage{indentfirst}
\usepackage{amsfonts}
\usepackage{amssymb}
\usepackage[left=2cm,right=2cm,top=2cm,bottom=2cm]{geometry}
\title{Функциональный анализ}

\newcommand{\overbar}[1]{\mkern 1.5mu\overline{\mkern-1.5mu#1\mkern-1.5mu}\mkern 1.5mu}

\theoremstyle{definition}
\newtheorem*{definition}{Определение}

\theoremstyle{plain}
\newtheorem*{theorem}{Теорема}

\theoremstyle{remark}
\newtheorem*{remark}{Замечание}

\theoremstyle{remark}
\newtheorem*{example}{Пример}

\theoremstyle{remark}
\newtheorem*{examples}{Примеры}

\theoremstyle{remark}
\newtheorem*{cexample}{Контр-пример}

\theoremstyle{plain}
\newtheorem*{lemma}{Лемма}

\theoremstyle{plain}
\newtheorem*{corollary}{Следствие}

\setcounter{tocdepth}{1}

\def\labelitemi{--}

\renewcommand{\qedsymbol}{\rule{0.7em}{0.7em}}

\begin{document}

\begin{titlepage}

\begin{center}
\vfill

Санкт-Петербургский государственный университет\\
\ \\

\vfill

{\large\bf ФУНКЦИОНАЛЬНЫЙ АНАЛИЗ\\}
\ \\
Лекции для студентов факультета ПМ-ПУ\\
(III курс, 6-ой семестр)

\vfill

\hfill\vbox
{
\hbox{Доцент кафедры моделирования электромеханических}
\hbox{и компьютерных систем, кандидат физ.-мат. наук}
\hbox{Владимир Олегович Сергеев}
}

\vfill

Санкт-Петербург, 2016
\end{center}

\end{titlepage}

\tableofcontents
\chapter{Сопряженное пространство. Сопряженный оператор}

\section{Сопряженное пространство}

Согласно определению пространством $X^*$, сопряженным нормированному пространству $X$, называется пространство линейных функционалов $f$, заданных на всем $X$. Все результаты, полученные для линейных операторов, переносятся на частный случай линейных функционалов.

Пространство $X^*$ --- полное пространство (пространство Банаха), $\lVert f\rVert = \sup\limits_{\lVert x\rVert = 1}\lvert f(x)\rvert$ (глава 2, $\S 2$).

В дальнейшем наряду с записью значения функционала $f$ элементы $x \in X$ мы будем обозначать
\begin{equation*}
f(x) = <x, f>
\end{equation*}
Такое обозначение имеет своё обоснование.
\begin{theorem}[Рисса]
(Фридьёф Рисс, 1880-1956 г., один из основоположников функционального анализа) \underbar{Об общем виде линейного функционала в пространстве Гильберта $H$.} Для любого $f \in H^*$ существует единственный элемент $y \in H$, такой что $<x, f>=(x,y)$ ($f(x)=(x,y)$, $(x,y)$ --- скалярное произведение в $H$).
\end{theorem}

\begin{proof}
Обозначим $L$ подпространство элементов $z$ таких, что значения функционала $f$ равно 0: $<z, f>=0$. Можно считать, что $L\ne H$ в противном случае $f(x)=0$ для любого $x\in H$, $\lVert f\rVert = 0$, $y=\circleddash$.

Ортогональное дополнение подпространства $L$ не пусто. Пусть $x_0\perp L$, тогда и элемент $\lambda x_0\perp L$, и можно считать, что $<x_0, f>=1$.

Пусть $x$ --- любой элемент $H$. На элементе $x-<x, f>x_0$ значение функционала $f$ равно:
\begin{equation*}
<x-<x, f>x_0, f> = <x, f>-<x, f><x_0, f> = <x, f>-<x, f> = 0
\end{equation*}

Следовательно элемент $x-<x, f>x_0 \subset L$, a $x_0\perp L$:

\begin{equation*}
(x-<x, f>x_0, x_0) = 0; (x, x_0) - <x, f>(x_0, x_0) = 0
\end{equation*}

Тогда $(x, f) = \frac{(x, x_0)}{\lVert x_0\rVert^2}$ и в качестве элемента $y$ можно взять элемент $y=\frac{x}{\lVert x_0\rVert^2}$:$<x, f> = (x, y)$.

Для нормы функционала $f$: $\lvert <x, f>\rvert \le \lVert y\rVert\lVert x\rVert$. Тогда $\lVert f\rVert \le \lVert y\rVert$. С другой стороны $<y, f> = (y, y) \le \lVert f\rVert\lVert y\rVert$, $\lVert y\rVert\le\lVert f\rVert$

Объединяя эти неравенства, получаем $\lVert f\rVert = \lVert y\rVert$.

\underbar{Единственность элемента $y$}: предположим, что существует другой элемент $y_1 \in H$ такой, что для любого $x\in H$:
\begin{equation*}
<x, f> = (x, y) = (x, y_1)
\end{equation*}
Тогда $(x, y-y_1) = 0$ и, взяв элемент $x = y-y_1$, получим $\lVert y-y_1\rVert = 0$, т.е. $y_1 = y$.
\end{proof}
Сходимости последовательности функционалов $\{f_n\}\in X^*$ к функционалу $f_0\in X^*$: \textbf{сильная сходимость}, если $\lVert f_n - f_0\rVert \to 0$ при $n\to\infty$; \textbf{поточечная сходимость} $f_n(x)\to f_0(x)$ при $n\to\infty$ для любых элементов $x\in X$. Верна теорема (Банаха-Штейнгауза, глава 2, \S 3):

Для того, чтобы последовательность $f_n$ сходилась поточечно к линейному функционалу, необходимо и достаточно, чтобы:

\begin{enumerate}
\item Нормы $f_n$ были ограничены в совокупности: $\lVert f_n\rVert \le const$.
\item Существуют пределы числовых последовательностей $f_n(x)$ при $n\to\infty$ для всех элементов $x$, принадлежащих множеству $D$
    всюду плотному в $X$.
\end{enumerate}

Введение сопряженного пространства приводит к новому типу сходимости последовательности $\{x_n\}$ элементов $x_n$ пространства $X$.
\end{document}